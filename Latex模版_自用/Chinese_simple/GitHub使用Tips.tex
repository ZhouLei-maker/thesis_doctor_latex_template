\PassOptionsToPackage{quiet}{fontspec}  %depress the problem of font
\documentclass[12pt, a4paper, oneside]{ctexart}
\usepackage{amsmath, amsthm, amssymb, bm, graphicx, hyperref, mathrsfs}

\title{\textbf{GitHub使用技巧}}
\author{周雷}
\date{\today}
\linespread{1.5}
\newcounter{problemname}
\newenvironment{problem}{\stepcounter{problemname}\par\noindent\textbf{题目\arabic{problemname}. }}{\\\par}
\newenvironment{solution}{\par\noindent\textbf{解答. }}{\\\par}
\newenvironment{note}{\par\noindent\textbf{题目\arabic{problemname}的注记. }}{\\\par}

\begin{document}

\maketitle

\begin{problem}
    初识Github。
\end{problem}

\begin{solution}
GitHub 是一个面向开源及私有软件项目的托管平台,因为只支持 Git 作为唯一的版本库格式进行托管,故名 GitHub。GitHub 于 2008 年 4 月 10 日正式上线,除了 Git 代码仓库托管及基本的 Web 管理界面以外,还提供了\emph{订阅、讨论组、文本渲染、在线文件编辑器、协作图谱(报表)、代码片段分享(Gist)}等功能。目前,其托管版本数量非常之多,而且其中不乏知名开源项目,例如 Ruby on Rails、jQuery、python 等。

作为开源代码库以及版本控制系统,Github 拥有超过千万的开发者用户。随着越来越多的应用程序转移到了云上,Github 已经成为了管理软件开发以及发现已有代码的首选方法。如前所述,作为一个分布式的版本控制系统,在 Git 中并不存在主库这样的概念,每一份复制出的库都可以独立使用,任何两个库之间的不一致之处都可以进行合并。GitHub 可以托管各种 Git 库,并提供一个 web 界面,但与其它像 SourceForge 或 Google Code 这样的服务不同,GitHub 的独特卖点在于从另外一个项目进行分支的简易性。为一个项目贡献代码非常简单:首先点击项目站点的Fork的按钮,然后将代码检出并将修改加入到刚才分出的代码库中,最后通过内建的pull request机制向项目负责人申请代码合并。GitHub 项目本身自然而然的也在 GitHub 上进行托管,只不过在一个私有的,公共视图不可见的库中。开源项目可以免费托管,但私有库则并非如此。在 GitHub,用户可以通过Explore轻而易举地找到海量的开源代码。因此,称之为程序员的 圣地 也不过吧?
\end{solution}

\begin{problem}
    维护自己的网页
\end{problem}

\begin{solution}
    如果以后我们要在 GitHub 上搭建自己的个人博客,其默认地址就是\href{https://github.com/ZhouLei-maker}{https://github.com/ZhouLei-maker}。GitHub 的仓库分为两种,一种是public repositories公开免费版,一种是private repositories私有付费版。其中,私有仓库一般是由企业或者不希望自己的仓库公开的个人用户购买,这也是 GitHub 的主要收入来源。
\end{solution}

\begin{problem}
    Github文件结构
\end{problem}

\begin{solution}
    包含 3 个commit,第一个 commit 是我们通过勾选Initialize this repository with a README,创建了一个初始化提交文件README.md,其中文件后缀为.md,表示文件为 Markdown 格式;包含 1 个branch,为master分支,即主分支;包含 1 个contributor,为贡献者,也就是我们自己。
\end{solution}

\begin{problem}
    Github常用术语
\end{problem}

\begin{solution}
    Repository:简称Repo,可以理解为“仓库”,我们的项目就存放在仓库之中。也就是说,如果我们想要建立项目,就得先建立仓库;有多个项目,就建立多个仓库。
    
    Issues:可以理解为“问题”,举一个简单的例子,如果我们开源一个项目,如果别人看了我们的项目,并且发现了bug,或者感觉那个地方有待改进,他就可以给我们提出Issue,等我们把Issues解决之后,就可以把这些Issues关闭;反之,我们也可以给他人提出Issue。
    
    Star:可以理解为“点赞”,当我们感觉某一个项目做的比较好之后,就可以为这个项目点赞,而且我们点赞过的项目,都会保存到我们的Star之中,方便我们随时查看。在 GitHub 之中,如果一个项目的点星数能够超百,那么说明这个项目已经很不错了。

Fork:可以理解为“拉分支”,如果我们对某一个项目比较感兴趣,并且想在此基础之上开发新的功能,这时我们就可以Fork这个项目,这表示复制一个完成相同的项目到我们的 GitHub 账号之中,而且独立于原项目。之后,我们就可以在自己复制的项目中进行开发了。

Pull Request:可以理解为“提交请求”,此功能是建立在Fork之上的,如果我们Fork了一个项目,对其进行了修改,而且感觉修改的还不错,我们就可以对原项目的拥有者提出一个Pull请求,等其对我们的请求审核,并且通过审核之后,就可以把我们修改过的内容合并到原项目之中,这时我们就成了该项目的贡献者。

Merge:可以理解为“合并”,如果别人Fork了我们的项目,对其进行了修改,并且提出了Pull请求,这时我们就可以对这个Pull请求进行审核。如果这个Pull请求的内容满足我们的要求,并且跟我们原有的项目没有冲突的话,就可以将其合并到我们的项目之中。当然,是否进行合并,由我们决定。

Watch:可以理解为“观察”,如果我们Watch了一个项目,之后,如果这个项目有了任何更新,我们都会在第一时候收到该项目的更新通知。

Gist:如果我们没有项目可以开源或者只是单纯的想分享一些代码片段的话,我们就可以选择Gist。不过说心里话,如果不翻墙的话,Gist并不好用。
\end{solution}

\begin{problem}
    国内类似的仓库
\end{problem}
\begin{solution}
    \href{https://gitee.com}{https://gitee.com}
    \\账号:18721290631
    \\密码:zhou0624
    \\用户名:ZhouLei\_Maker
    \\绑定邮箱:\href{18240327963@163.com}{18240327963@163.com}
\end{solution}

\begin{problem}
    git操作步骤:
\end{problem}

\begin{solution}
    步骤:
    \begin{itemize}
        \item 在目标文件夹中打开终端窗口,初始化文件夹,git init 
        \item 将文件夹中的文件添加进缓冲区,git add . "."表示将所有文件夹都添加进去
    \end{itemize}
\end{solution}

\end{document}
\documentclass{MyLatex}
\setlength{\headheight}{39pt}
\begin{document}

% 标题,作者
\emptitle{航空发动机简介}
\empauthor{周雷}{}{徐弘一}
% 奇数页页眉 % 请在这里写出第一作者以及论文题目
\fancyhead[CO]{{\footnotesize 周雷: 航空发动机}}


%%%%%%%%%%%%%%%%%%%%%%%%%%%%%%%%%%%%%%%%%%%%%%%%%%%%%%%%%%%%%%%%
% 关键词 摘要 首页脚注
%%%%%%%%关键词
\Keyword{航空发动机}
\twocolumn[
\begin{@twocolumnfalse}
\maketitle

%%%%%%%%摘要
\begin{empAbstract}本文主要介绍航空发动机的结构和制造公司等杂项。

\end{empAbstract}

%%%%%%%%首页角注,依次为实验时间、报告时间、学号、email
\empfirstfoot{2022-10-12}{2022-10-12}
\end{@twocolumnfalse}
]
%%%%%%%%!首页角注可能与正文重叠,请通过调整正文中第一页的\enlargethispage{-3.3cm}位置手动校准正文底部位置:
%%%%%%%%%%%%%%%%%%%%%%%%%%%%%%%%%%%%%%%%%%%%%%%%%%%%%%%%%%%%%%%%
%  正文由此开始
\wuhao 
%  分栏开始

\section{航空发动机制造公司}

以下是一些全球知名的航空发动机制造公司:
\begin{itemize}
  \item GE Aviation - 美国通用电气旗下的航空发动机制造公司,生产各种商用和军用航空发动机,包括CFM56、GE90、GEnx等。
  \item Rolls-Royce - 英国劳斯莱斯旗下的航空发动机制造公司,生产各种商用和军用航空发动机,包括Trent、RB211、Adour等。
  \item Pratt \& Whitney - 美国普惠旗下的航空发动机制造公司,生产各种商用和军用航空发动机,包括PW4000、PW2000、F119等。
  \item Safran Aircraft Engines - 法国赛峰航空发动机公司,生产各种商用和军用航空发动机,包括CFM56、LEAP、M88等。
  \item MTU Aero Engines - 德国MTU航空发动机公司,生产各种商用和军用航空发动机,包括PW1000G、TP400、EJ200等。
\end{itemize}
除了以上这些知名的航空发动机制造公司,还有一些其他的制造商,例如Honeywell Aerospace (霍尼韦尔)、IAE(航空发动机制造联合企业)、IHI(日本航空宇宙系统公司)等。 \enlargethispage{-3.3cm}

\section{航空发动机主要部件}


\begin{itemize}
  \item 压气机:负责将空气压缩,提高空气压力和温度,以增加燃油的燃烧效率。
  \item 燃烧室:将燃料和压缩后的空气混合并点燃,产生高温高压气体,提供动力驱动涡轮。
  \item 涡轮:由高温高压气体驱动,带动压气机和风扇等部件运转,产生动力。
  \item 推力矢量控制(TVC):通过改变排气喷口的方向和位置,实现发动机喷气流的调节,调节飞机的方向和姿态。
  \item 空气滤清器:过滤进入发动机的空气,防止杂质和颗粒物损坏发动机部件。
  \item 空气冷却器:对发动机内部的高温部件进行冷却,保持发动机正常工作温度。
  \item 润滑系统:为发动机提供润滑油,减少部件磨损和摩擦,延长使用寿命。
  \item 点火系统:点燃混合物,开始燃烧过程。
  \item 排气系统:将燃烧产生的废气排出发动机,并通过喷气产生推力。
  \item 控制系统:对发动机的运行参数进行监控和调节,确保发动机的正常工作和高效运行。
\end{itemize}

在流体力学研究范畴内,关注的是压气机和涡轮,可能也会接触一些燃烧室。

\section{常见发动机}

\emph{CFM56}:是由美国通用电气和法国赛峰集团联合开发的一款高涵道比双轴涡扇发动机,广泛应用于\emph{中短程客机和货机}。

\emph{V2500}:是一款由国际航空发动机公司(IAE)联合研发的双轴涡扇发动机,被广泛应用于\emph{空客A320系列客机}。

\emph{PW4000}:是由普惠公司研发的一款高涵道比双轴涡扇发动机,广泛应用于\emph{波音747、767、777等宽体客机}。

\emph{RB211}:是由劳斯莱斯公司研发的一款高涵道比三轴涡扇发动机,广泛应用于\emph{波音747和英国航空A380客机}。

\emph{GE90}:是由通用电气公司研发的一款高推力涡扇发动机,被广泛应用于\emph{波音777系列客机}。

\emph{Trent 1000}:是由劳斯莱斯公司研发的一款高涵道比双轴涡扇发动机,被广泛应用于\emph{波音787梦幻客机}。

\section{压气机性能指标}

航空发动机中的压气机性能指标包括以下几个方面:

\emph{压气机效率}(Compressor Efficiency):指压气机转换空气动能为压力能的效率,一般用压比(Pressure Ratio)表示,即压气机出口总压与入口总压之比。压比越大,效率越高,同时也能减少排放。

\emph{压气机压比}(Pressure Ratio):指压气机出口总压与入口总压之比,是评价压气机性能的重要指标。压比越大,发动机的推力和效率越高,但会导致温度上升和材料受热损伤等问题。

\emph{压气机流量}(Airflow):指单位时间内通过压气机的气体体积或质量,是评价压气机输出功率和推力的重要指标。压气机的流量大小决定了发动机的功率和推力大小。

\emph{压气机转速}(Rotational Speed):指压气机转子的旋转速度,是影响压气机流量和压比的重要因素。转速越高,压气机的流量和压比也会相应增加,但同时也会产生更多的机械压力和摩擦损耗。

\emph{压气机质量流量}(Mass Flow Rate):指单位时间内通过压气机的气体质量,是评价压气机输出功率和推力的重要指标。质量流量越大,发动机的推力也会相应增加。

\emph{压气机功率}(Compressor Power):指压气机输出的功率,包括轴功和气功等部分。

综上所述,压气机性能指标的主要目标是提高效率、增加流量、降低燃油消耗和减少排放。这些指标之间相互影响,需要在实际设计和使用中做出平衡考虑。

\section{涡轮性能指标}

航空发动机涡轮部件的性能指标主要包括以下几个方面:

\emph{转速}(Rotational Speed):指涡轮转子的旋转速度。涡轮的转速需要满足发动机的功率和推力需求,同时也要考虑到涡轮的强度和稳定性等因素。

\emph{转矩}(Torque):指涡轮转子输出的扭矩。涡轮的转矩需要满足发动机的功率和推力需求,同时也要考虑到涡轮的强度和耐久性等因素。

\emph{效率}(Efficiency):指涡轮转换热能为动能的效率。涡轮的效率越高,燃油消耗就越低,同时也能减少排放。

\emph{耐久性}(Durability):指涡轮在长期运行过程中能够保持其性能和寿命的能力。涡轮需要具备足够的强度和抗疲劳性能,以确保其在高温、高压和高转速等极端条件下能够长期稳定运行。

\emph{叶片数}(Number of Blades):指涡轮转子上的叶片数目。叶片数目的选择需要考虑到转速、转矩、效率和噪音等因素,以实现最佳性能。

\emph{材料}(Materials):涡轮部件的材料需要具备高温、高压和高转速下的耐久性能,并且要轻量化、耐腐蚀和易于制造。

综上所述,航空发动机涡轮部件的性能指标需要满足高效率、耐久性和轻量化等要求,以确保发动机在各种工况下都能够稳定、可靠地运行,并且能够满足航空运输业的要求。


%\subsection{报告中表的规范}
%推荐使用标准“三线表”(如表\ref{tab:eg1}所示。),内容易混淆时可加辅助线进行辅助说明。按表格在文中出现的顺序,用阿拉伯数字对其进行编号,全文顺序编号。应%有相应的表题且每个表格前都应有相应的引出或介绍文字。

%图、表应在文中有相应表述,即图、表的号应在文中引出,以先见文后见图、表为原则。每个图、表都必须有图名、表名,并且有编号。图号、表号应全文统一连续排列,%即,应按照图1、图2……排列不应按小结编号。图片中的文字、线条应当清晰可辨,图片像素(DPI)在300以上。
%表格推荐采用全线表,表头中使用量符号/量单位形式。如表\ref{tab:eg2}所示。

% \begin{table}[H]
% \centering
% \captionnamefont{\wuhao\bf\heiti}
% \captiontitlefont{\wuhao\bf\heiti}
% \caption{三线表示例} \label{tab:eg1}
% \liuhao
% \begin{tabular}{cccc}
% \toprule
% {编号} &  {直径}/\si{\metre} & {静温}/\si{\kelvin} & {时间}/min\\
% \midrule 
% 4 & 0.0349 & 268.15 & 30\\
% 5 & 0.01905 & 268.15 & 30\\
% \bottomrule
% \end{tabular}
% \end{table}

%\begin{table}[h]
%\centering
%\captionnamefont{\wuhao\bf\heiti}
%\captiontitlefont{\wuhao\bf\heiti}
%\caption{全线表示例} \label{tab:eg2}
%\liuhao
%\begin{tabular}{|c|c|c|c|c|}
%\hline
%U/V & I/mA & v/km·h$^{-1}$ & x/mm & p/MPa \\ \hline
%\textit{12} & \textit{30} & \textit{80} & \textit{55} & \textit{110} \\ \hline
%\textit{24} & \textit{34} & \textit{90} & \textit{60} & \textit{111} \\ \hline
%\end{tabular}
%\end{table}

%\subsection{报告中英文缩略语的规范}



%\subsection{外文字母}
%\subsubsection{斜体外文字母用于表示量的符号,主要用于下列场合}

%\begin{enumerate}
%\renewcommand{\labelenumi}{(\theenumi)}
%\item 变量符号、变动附标及函数。
%\item 用字母表示的数及代表点、线、面、体和图形的字母。
%\item 特征数符号,如Re (雷诺数)、Fo (傅里叶数)、Al (阿尔芬数)等。
%\item 在特定场合中视为常数的参数。
%\end{enumerate} 


%\subsubsection{正体外文字母用于表示名称及与其有关的代号,主要用于下列场合}
%\begin{enumerate}
%\renewcommand{\labelenumi}{(\theenumi)}
%\item 有定义的已知函数(例如$\sin$, $\exp$, $\ln$等)。
%\item 其值不变的数学常数(例如$\mathrm{e} = 2.718 281 8\cdots)$及已定义的算子。
%\item 法定计量单位、词头和量纲符号。
%\item 数学符号。
%\item 化学元素符号。
%\item 机具、仪器、设备和产品等的型号、代号及材料牌号。
%\item 硬度符号。
%\item 不表示量的外文缩写字。
%\item 表示序号的拉丁字母。
%\item 量符号中为区别其他量而加的具有特定含义的非量符号下角标。
%\end{enumerate} 

%\section{结~~论}



%%%%%%%%%%%%%%%%%%%%%%%%%%%%%%%%%%%%%%%%%%%%%%%%%%%%%%%%%%%%%%%%
%  参考文献
%%%%%%%%%%%%%%%%%%%%%%%%%%%%%%%%%%%%%%%%%%%%%%%%%%%%%%%%%%%%%%%%
%  参考文献按GB/T 7714-2015《文后参考文献著录规则》的要求著录. 
%  参考文献在正文中的引用方法:\cite{bib文件条目的第一行}

%\renewcommand\refname{\heiti\wuhao\centerline{参考文献}\global\def\refname{参考文献}}
%\vskip 12pt

%\let\OLDthebibliography\thebibliography
%\renewcommand\thebibliography[1]{
  %\OLDthebibliography{#1}
 % \setlength{\parskip}{0pt}
%  \setlength{\itemsep}{0pt plus 0.3ex}
%}

%{
%\renewcommand{\baselinestretch}{0.9}
%\liuhao
%\bibliographystyle{gbt7714-numerical}
%\bibliography{./TempExample}
%}



\end{document} 
